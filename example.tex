% !Mode:: "TeX:UTF-8"
% !TEX program  = xelatex

\documentclass{cumcmthesis}
%\documentclass[withoutpreface,bwprint]{cumcmthesis} %去掉封面与编号页,电子版提交的时候使用。


\usepackage[framemethod=TikZ]{mdframed}
\usepackage{url}   % 网页链接
\usepackage{subcaption} % 子标题
\title{全国大学生数学建模竞赛编写的 \LaTeX{} 模板}
\tihao{C}
\baominghao{202006010326}
\schoolname{大连理工大学}
\membera{}
\memberb{}
\memberc{}
\supervisor{ }
\yearinput{2020}
\monthinput{09}
\dayinput{13}

\begin{document}

 \maketitle
 \begin{abstract}
    本文根据中小微企业票据信息建立全连接神经网络模型预测企业违约率,完成信贷风险的量化分析,然后建立最优化模型分别求出每个企业的贷款额度和年利率,最后通过层次分析法模型完成企业实力的量化分析,对突发因素下的信贷策略进行修正。
    
    针对问题一,要求对信贷风险进行量化分析并给出银行的信贷策略。我们首先对附件一的发票数据进行了预处理,提取出1200000个$8 \times 5$的特征矩阵并附上是否违约的标签,作为全连接网络的输入数据实现二分类模型,附件一中的27个违约企业我们成功预测了其中13个。对于信贷策略的制定实际上是确定每一个企业的信贷额度和年利率,首先根据信贷风险按比例划分银行的信贷总额求出每一个企业的信贷额度,然后确定年利率的优化目标是使银行年收益更大和客户流失更少,据此我们根据企业的信誉等级求得该等级下的基准年利率,然后按照企业违约率进行修正从而得到最终的年利率。
    
    针对问题二,我们运用问题一中的数据预处理方式和全连接网络模型得到每一个企业的预测违约率,然后根据附件一中的信誉等级与违约率的关系建立模型,从而确定附件二302家企业的信誉等级,最后按照与问题一相同的方法制定信贷策略。
    
    针对问题三,要求我们对突发因素下的信贷策略进行调整,实际上只需对之前模型的违约率进行人工微调即可。首先根据突发因素的影响行业确定违约率的修正方向,然后根据所处行业,票据信息的统计量建立基于层次分析法的企业实力量化模型,接着根据企业实力和突发因素的剧烈程度确定违约率的修正范围,最后按照与问题二相同的方法制定信贷策略。
    
    本文运用层次分析法建立企业实力分析的量化模型,将复杂,模糊的问题量化讨论。另外针对数据预处理时的数据分布不均衡和数值超过程序处理范围的问题,创新性地提出了重采样和归一化的方法,并运用全连接神经网络去挖掘票据信息中的隐藏特征,避免了人工分析的繁琐。
    

\keywords{全连接网络\quad  最优化\quad   层次分析法 \quad  数据预处理}
\end{abstract}

%目录  2019 明确不要目录,我觉得这个规定太好了
%\tableofcontents

%\newpage

\section{背景介绍}

众所周知,信息不透明、信息不对称是中小企业融资难的基本原因。与大企业相比,中小企业由于难以提供经审核合格的财务资料和经营记录,在向银行贷款时面临着更高的信贷风险。为了防止信息不对称造成的逆向选择和道德风险,银行往往会实施信贷配给政策,有选择地对有信贷需求的企业进行授信,拒绝不能满足信息要求的借款人的申请,国外大量的经验证据表明,信贷配给的主要对象正是信息不透明的中小企业。然而商业银行为了尽可能的减少信息不对称,会加大监督执行成本,而且在收集和处理信息方面较之中小企业而言是规模而不经济的,直接增加商业银行对中小企业的贷款成本,商业银行不愿对缺乏相应信息的企业提供信贷支持而产生了“惜贷”的现象。

为了有效缓解金融交易中的信息问题,银行根据不同类型的信息制定了多种贷款技术。美国经济学家Berger等人将其分为四大类:财务报表信贷、资产支持信贷、信用评分信贷和关系型信贷。由于中小微企业融资具有需求多、额度小、期限短、时效性强、时间急、贷款频率高的特点,这种状况的形成与中小微企业自身、银行偏好“批发信贷”和政策因素紧密相关,信用评分技术在中小企业信贷策略的制定中应用极为广泛 。

信用评分技术是利用现代数理统计模型和信息技术,对客户的信用记录进行测算和分析,从而做出决策的新技术。它具有成本低、效率高的特点,可以在很大程度上缓解信息问题,从20世纪80年代开始,美国金融机构就开始使用这种技术发放小额贷款。但由于该技术的复杂性,对信息系统和数据积累的要求较高,其应用受到限制。		

解决中小微企业融资难的问题对于不断完善金融服务,拓宽中小微企业融资渠道,积极支持中小微企业发展壮大、对建立健全有效的融资机构都有着十分重要的理论和现实意义。考虑到本题可供分析的数据比较单一且常用的信用评分技术较为复杂,为银行提供一个简单有效的信贷策略制定模型显得尤为重要,因此我们借鉴了财务报表技术和信用评分技术的核心思想,充分挖掘票据信息等已有数据中的关系,并结合国家经济体制改革的政策,期望让银行以较低的贷款成本完成对中小微企业信贷策略的制定。



\section{问题重述}

在实际中,由于中小微企业规模相对较小,也缺少抵押资产,因此银行通常是依据信贷政策、企业的交易票据信息和上下游企业的影响力,向实力强、供求关系稳定的企业提供贷款,并可以对信誉高、信贷风险小的企业给予利率优惠。银行首先根据中小微企业的实力、信誉对其信贷风险做出评估,然后依据信贷风险等因素来确定是否放贷及贷款额度、利率和期限等信贷策略。

某银行对确定要放贷企业的贷款额度为10\textasciitilde 100万元;年利率为4\% \textasciitilde 15\%;贷款期限为1年。已经给出了123家有信贷记录企业的相关数据、302家无信贷记录企业的相关数据和贷款利率与客户流失率关系的2019年统计数据。该银行希望根据实际和已给的数据信息,通过建立数学模型研究对中小微企业的信贷策略,主要解决下列问题:
\begin{itemize}
    \item 123家有信贷记录企业的信贷风险进行量化分析,给出该银行在年度信贷总额固定时对这些企业的信贷策略。
    \item 在问题1的基础上,对302家302家无信贷记录企业的信贷风险进行量化分析,并给出该银行在年度信贷总额为1亿元时对这些企业的信贷策略。
    \item 企业的生产经营和经济效益可能会受到一些突发因素影响,而且突发因素往往对不同行业、不同类别的企业会有不同的影响。综合各企业的信贷风险和可能的突发因素(例如:新冠病毒疫情)对各企业的影响,给出该银行在年度信贷总额为1亿元时的信贷调整策略。
\end{itemize}

\section{问题分析}
\subsection{问题一的分析}
问题一首先需要我们对123家有信贷记录企业的信贷风险进行量化分析。由于信贷风险主要集中在中小微企业可能发生的违约上,所以对信贷风险的量化分析实际上可以转化为对企业违约率的分析上。由于附件一已经给出了123家有信贷记录企业是否违约的数据,因此我们考虑抽取出进项发票信息和销项发票信息等数据特征,并结合是否违约的标签训练一个二分类器,从而预测企业的违约率。由于原始数据分布不均匀,数值差异化很大,因此需要完善有效的数据预处理方案。考虑到输入数据的可解释性差,所以基于线性模型或者树模型的人工建模的方案可能不太适用,为了充分挖掘输入数据中的隐藏信息,我们选择了全连接神经网络。

问题一其次需要我们对这些企业制定在年度信贷总额固定时的信贷策略。由于题目已经确定贷款期限为1年,所以我们不必考虑信贷策略中的贷款期限因素,后续我们的银行收益计算也是以一年为限度。我们将会给出的信贷策略包含两个方面:贷款额度、年利率。根据题目信息和生活经验,贷款风险低信誉评级高的企业应该分配更大的信贷额度,并为其贷款年利率给予一定的优惠。考虑到了信誉评级为D的企业不予贷款,因此我们需要对剩下的企业按照他们的预测违约率对银行的贷款总额进行划分,满足违约率低的企业将会得到更多的贷款额度。然后在每家企业贷款额度确定的情况下寻找最优的年利率,这可以转换成关于年利率的最优化问题,最优化目标是企业的年收益最大和客户流失率最小。求得每一个企业的贷款额度和年利率即为最终的贷款策略。

\subsection{问题二的分析}
问题二相较于问题一的差别主要是给出的302家企业没有贷款记录,并且信贷总额被限制在1亿元,基本思路和问题一保持一致。首先运用问题一中的数据预处理方式得到模型的输入特征矩阵和对应的标签,喂入全连接网络模型后得到每一个企业的预测违约率。与问题一不同的是,这302家企业的信誉评级数据缺失,所以我们先要根据附件一中的信誉等级与违约率的关系建立模型,然后对这302家企业的信誉评级进行补全,最后按照与问题一相同的方法确定每一个企业的贷款额度和年利率即为最终的贷款策略。
\subsection{问题三的分析}
首先补充说明我们后续会用到的对企业实力的量化分析方案,我们考虑进项和销项发票信息中价税合计的统计量(最大值,均值,和)以及企业是否处于社会热门行业共7个因素,并建立层次分析法模型得到302家企业的量化实力权重向量,向量值越大的企业代表相对实力越强。

问题三要求对突发因素下的信贷策略进行调整,使模型更加接近于真实情况,实际上只需对之前模型的预测违约率进行人工微调即可。我们可以首先根据突发因素的影响行业确定违约率的修正方向,不同类型的企业修正方向是不同的,如疫情的爆发等对医疗有关企业有促进作用,而对食品有关企业有抑制作用;然后根据企业实力和突发因素的剧烈程度确定违约率的修正范围,一般来说,综合实力强的企业受突发因素的影响较小,综合实力弱的企业受突发因素的影响较大,另外如果突发因素客观上非常剧烈,对社会影响很大,那么对应的违约率的修正范围也会相应的增大。最后使用微调过的预测违约率按照与问题二相同的方法确定最终的信贷策略。


\section{模型假设以及符号说明}
\subsection{模型假设}
\begin{itemize}
    \item 假设所有数据来源真实可靠,不存在数据质量问题。
    \item 假设附件一和附件二的数据分布相同,以便我们根据使用附件一训练的模型有效预测附件二的企业的违约率。
    \item 假设银行给与中小微企业的信贷额度都能被企业完全使用,以便我们计算信贷策略确定时的银行年收益。
    \item 假设客户流失率仅仅与银行的年利率设置有关。
    \item 假设客户的信誉评级仅仅与客户的违约概率有关。
\end{itemize}
\subsection{变量及常量符号说明}

\begin{table}[H] 
    
    \label{tablesymbol}
    \centering
    \begin{tabular}{c|c}   
    \hline
    变量或常量符号 & 变量或常量符号解释 \\
    \hline 
    $P_i$ & 企业i的违约率 \\
    $Q_i$ & 企业i的信誉评级,可选取值为A,B,C,D \\
    $\beta_k$ & 信用评级为k的企业的基准年利率 \\
    $\rho_i$ & 企业i的年利率    \\
    $M_i$ & 企业i的信贷额度 \\
    $\theta_i$ & 企业i的实力权重 \\
    $N$ & 银行每年平摊在每笔贷款上的固定成本 \\
    $K$ & 银行的信贷总额 \\
    $W$ & 银行从借贷中获得的总收益 \\
    $\alpha$ & 突发因素对违约率的基准影响程度 \\



    \end{tabular}
    \caption{文中出现变量与常量符号及其解释}
    \end{table}
\section{模型的建立与求解}
\subsection{问题一数据预处理}
数据预处理的目标是从附件一抽取出进项发票信息和销项发票信息中的数据特征,构造数量足够的特征矩阵,并为每一个矩阵贴上是否违约的标签,作为输入数据训练一个二分类器。

我们面临的挑战包括:数据分布不均衡,即附件一中没有违约的企业数量远远大于有违约记录的企业数量;数值差异巨大,对于金额,税额,价税合计的取值差异很大,以至于均值归一化后计算机无法表示(只能表示为NaN)。对此我们运用了重采样和正态归一化的方法。具体做法如下:

\begin{itemize}
    \item  对附件一的企业信息,进项发票信息和销项发票信息进行表连接。然后,‘是否违约’这一项中用0表示否,用1表示是。‘发票状态’这一项用1表示正常发票,用0表示作废发票。添加一列‘是否进项’,并用1表示该行为进项发票信息,0为销项发票信息。
    \item 添加一列'是否热门',通过查阅资料,我们得知现阶段我国的贷款政策是:积极支持农副产品采购和扩大商品流通;大力支持生产出口创汇率的产品; 支持企业开发试制新产品,进行技术更新、技术改造和引进先进技术;支持科学技术为生产和商品流通服务;支持集体和个体企业的发展。从而确定了热门行业,并给出了热词:'控制','计算机','软件','硬件','网络','智能','IT','系统集成','电子','通信','设备','运营','科技','贸易','商贸','咨询','财会','法律','房地产','建筑','工程','机械制造','机电','重工','基金','证券','期货','投资’。如果企业名称中含有热词,那么‘是否热门’这一项为1,否则为0,这一步得到的结果示例如图\ref{fig51}。
    \begin{figure}[H]
        \centering
        \includegraphics[width=0.7\textwidth]{Figure5-1}
        \caption{企业发票信息的特征矩阵}
        \label{fig51}
    \end{figure}
    \item 为了整理出神经网络训练需要的等大小特征矩阵并解决数据分布不均衡的问题,这一步我们采用了重采样的方法对每一个违约的企业生成了16000个特征矩阵,对每一个没有违约的企业生成了8000个特征矩阵,并贴上是否违约的对应标签。每个特征矩阵生成的策略是从企业的所有数据行中随机选择8行,并保留‘是否热门,金额 ,税额,发票状态,是否进项’共5列信息,组成一个$8 \times 5$的矩阵。这一步产生的结果示例如图\ref{fig52}。
    \begin{figure}[H]
        \centering
        \includegraphics[width=0.7\textwidth]{Figure5-2}
        \caption{重采样后的特征矩阵}
        \label{fig52}
    \end{figure}
    \item 可以看到金额和税额这两列的值差异巨大,如果直接采用最简单的均值归一化方法,精度会超出计算机的表示范围,程序会无法识别,只好显示为NaN。这里采用的是正态归一化,采用$\mu$表示数据列的均值,用$\sigma$表示数据列的标准差,正态归一化的表达式为
    \begin{equation}
        x_i \gets \frac{x_i-\mu}{\sigma}
        \label{equa1}
    \end{equation}
    另外我们还对超出$\mu \pm 3\sigma$的数据进行了截断,使其分布情况更加合理。这一步得到的结果示例如图\ref{fig53}。
    \begin{figure}[H]
        \centering
        \includegraphics[width=0.7\textwidth]{Figure5-3}
        \caption{正态归一化后的特征矩阵}
        \label{fig53}
    \end{figure}
    \item 共计整理出1200000个不同标签的特征矩阵,其中标签为1即有违约记录的企业有432000个特征图,数据量很大而且分布较为均匀,模型应该不会出现过共计整理出1200000个不同标签的特征矩阵,其中标签为1即有违约记录的企业有432000个特征图,数据量很大而且分布较为均匀,模型应该不会出现过拟合现象。
\end{itemize}

\subsection{问题一信贷风险量化分析}
由于信贷风险主要集中在中小微企业可能发生的违约上,所以对信贷风险的量化分析实际上可以转化为对企业违约率的分析上。首先根据前文数据预处理部分完成输入数据的整理,考虑到输入数据的可解释性差,所以我们选择了神经网络这种提取信息能力更强的模型。由于输入的矩阵为$8 \times 5$,所以使用简单的全连接网络足矣,没有必要使用卷积神经网络,网络结构如图\ref{fig54}。

\begin{figure}[H]
    \centering
    \includegraphics[width=0.7\textwidth]{Figure5-4}
    \caption{违约率预测的全连接神经网络}
    \label{fig54}
\end{figure}

网络一共四层,每层的神经元个数依次为8,6,4,1,第一层和第二层后面用relu函数(\ref{equarelu})进行激活,第三层后面接sigmoid激活函数(\ref{equasigm})输出一个0-1的概率值,将其与真实标签进行比较,算出误差并通过反向传播完成网络权重的更新,在训练200个epoch后模型正确率达到了64\%左右。
\begin{equation}
    \Phi(x) = max(0,x)
    \label{equarelu}
\end{equation}
\begin{equation}
    \Phi(x) = \frac{1}{1+e^{-x}}
    \label{equasigm}
\end{equation}

我们将从附件一中整理的1200000个特征矩阵喂入全连接神经网络预测得到输出概率,并对每一个企业的所有输出概率求均值,该值即为企业的预测违约率Pi,该值的大小可以衡量贷款风险的大小,每个企业的预测违约率Pi我们在附件一中具体给出。

\subsection{问题一信贷策略的制定}
我们还需要对这些企业制定在年度信贷总额固定时的信贷策略。根据上文问题一分析中的解释,我们将会给出每个企业的贷款额度、年利率作为该银行的信贷策略。
\subsubsection{信贷额度的确定}
首先是信贷额度Mi的确定,根据题目信息和生活经验,贷款风险低信誉评级高的企业应该分配更大的信贷额度,因此考虑使用上文预测的企业违约率对银行的信贷总额进行分配,其中信誉评级为D的企业不予分配贷款额度,满足违约率低的企业将会得到更多的贷款额度,具体的分配方式如式\ref{equalong1}。
\begin{equation}
\begin{split}
M_i=&
\begin{cases}
\frac{1-P_i}{P_{sum}}\times K, & Q_i \ne 'D'\\
0 & Q_i = 'D' \\
\end{cases} \\
\mbox{其中} &  P_{sum} = \sum_{j=1}^{123} a_j (1-P_j) \\
 a_j =& 
\begin{cases}
1 & Q_j \ne 'D' \\
0 & Q_j = 'D'\\
\end{cases}
\end{split}
\label{equalong1}
\end{equation}
我们首先求出某个信誉评级不为D的企业的信誉率占所有信誉评级不为D的企业的信誉率之和的比例,其中信誉率为$1-P_i$,然后根据这一比例划分银行的信贷总额,注意对不满足贷款额度为10\textasciitilde 100万元的值进行截断,最终可以得到每个企业的贷款额度$M_i$。
\subsubsection{年利率的确定}
在信贷额度确定的情况下,年利率的确定$\rho_i$可以看作一个最优化问题,优化目标是是企业的年收益最大和客户流失率最小。根据题目信息和生活经验,贷款风险低信誉评级高的企业应该给予一定的年利率优惠。综合分析后,我们的策略是先根据最优化目标求得特定信誉评级下的基准年利率,然后按照企业的预测违约率进行微调,当然信誉评级为D的企业没必要考虑年利率的取值,并且我们仅考虑影响年收益和客户流失率的众多影响因素中的年利率设置因素。


首先求特定信誉评级下的基准年利率。优化目标是是银行的年收益最大和客户流失率最小。银行在借贷过程中的主要收益是来自于借贷后产生的利息,但是每笔钱还需要扣除银行的平均运营成本。客户的流失率我们在假设中写到其仅仅与年利率的确定有关,所以相关的值直接查附件三即可得到。我们的优化目标可以表示为
\begin{equation}
    \begin{split}
    & \max \sigma(M_i \rho_i) - \sigma(F_j(\rho_i))
    \\
    \mbox{其中} &  \sigma(x_j) = \frac{\sum_{i=1}^{j}x_i}{\sum_{i=1}^{n}x_i} \\
    \end{split}
    \label{equalong2}
\end{equation}
式子中,$\rho_i$ 为年利率,$M_i$为企业i的信贷额度,$N$为银行每年的固定支出,
$F_j(\rho_i)$表示年利率为$\rho_i$,信誉评级为j时的客户损失率,$\sigma$表示归一化函数,其目的是为了消除银行年收益和客户流失率之间的量纲差异影响。前面一项表示的是归一化后的银行年收益,目标是使其最大,后面一项是归一化后的客户流失率,目标是使其最小,中间用减号连接,目标转化为合式最大。

接下来我们开始求解上述优化问题,根据附件3中可以发现,随着年利率的增大客户流失率逐渐增大,同等年利率下信誉评级为A的企业客户流失率始终最大。接着通过细致地考虑发现,由于归一化函数的作用,容易推导出最优化目标可以忽视Mi和N影响。因此我们只需要首先按照上面的归一化函数将年利率,信誉评级为A,B,C的客户损失率映射到同一个量纲的空间,然后在这个找到能够使得年利率和信誉评级为A,B,C的客户损失率的差值最大的年利率,求得的这三个年利率分别为信誉评级为A,B,C的基准年利率


\subsection{问题二模型建立与求解}
问题二相较于问题一的差别主要是给出的302家企业没有贷款记录,并且信贷总额被设置为1亿元,基本思路和问题一保持一致。首先运用问题一中的数据预处理方式得到模型的输入特征矩阵和对应的标签,喂入全连接网络模型后得到每一个企业的预测违约率。与问题一不同的是,这302家企业的信誉评级数据缺失。

我们这里介绍一下这302家企业的信誉评级的补全策略。首先提取出附件一中企业的信誉等级与违约率的关系,然后对提取出的表按照违约率排序,然后通过人工观察的方式确定三个合适违约率作为分位点,通过分位点划分即可确定企业的信誉评级,满足违约率越小的企业信誉等级越高。我们对附件二中的302家企业的信誉评级进行了补全并保存在了附件中,这里展示部分结果如下:

对这302家企业的信誉评级进行补全后,最后按照与问题一相同的方法确定每一个企业的贷款策略,我们将求得的每个企业的信贷额度和年利率保存在了附件中,这里展示部分结果如下:

\subsection{问题三企业实力量化分析}
我们后续的建模会用到的对企业实力的量化分析,所以这里首先介绍我们的企业实力量化分析方案。我们考虑进项和销项发票信息中价税合计的统计量(最大值,均值,和)以及企业是否处于社会热门行业共7个因素,建立层次分析法模型得到302家企业的量化实力权重向量,向量值越大的企业代表相对实力越强。层次分析法(AHP)是对一些较为复杂、较为模糊的问题作出决策的简易方法,它特别适用于那些难于完全定量分析的问题。它是美国运筹学家T. L. Saaty 教授于上世纪70 年代初期提出的一种简便、灵活而又实用的多准则决策方法。具体的方案如下。



\subsubsection{层次定义及判断矩阵构造}
首先要把问题条理化、层次化,构造出一个有层次的结构模型,对于本题,我们选择了三层分析对象:目标层O:选择综合实力最强的企业;准则层C:选取了7项作为决定因素;措施层P:附件二中302家企业。其中决定因素的选取我们通过分析票据数据信息并查阅资料最终确定为:进项发票价税合计的最大值,进项发票价税合计的总值,进项发票价税合计的均值,销项发票价税合计的最大值,销项发票价税合计的总值,销项发票价税合计的均值,该行业是否为热门行业共7项,分别记为$\left[ in\_max,in\_sum,in\_avg,out\_max,out\_sum,out\_avg,ishot \right] $,后续描述我们会使用简称。
层次分析法的三层设置如图\ref{MultiLayer}。
\begin{figure}[H]
    \centering
    \includegraphics[width=0.7\textwidth]{MultiLayer}
    \caption{层次分析法的三层设置}
    \label{MultiLayer}
\end{figure}

然后构造因素的判断矩阵,层次结构反映了因素之间的关系,但准则层中的各准则在目标衡量中所占的比重并不一定相同,在决策者的心目中,它们各占有一定权重。在确定影响某因素的诸因子在该因素中所占的比重时,遇到的主要困难是这些比重常常不易定量化。 为此Saaty 等人建议可以采取对因子进行两两比较建立成对比较矩阵的办法。设现在要比较 n 个因子 $X = \left\{ x_1,L, x_n \right\} $ 对某因素 $Z$ 的影响大小 ,即每次取两个因子 $x_i$ 和 $x_j$ ,以 $a_{ij}$ 表示 $x_i$ 和 $x_j$ 对因素$Z$的影响大小之比,全部比较结果用矩阵$A = (a_{ij} )n \times n $表示,称 A 为 Z - X 之间的成对比较判断矩阵(简称判断矩阵)。容易看出,若 $x_i$ 与 $x_j$ 对 Z 的影响之比为 $a_{ij}$ ,则 $x_j$ 与 $x_i$ 对 $Z$ 的影响之比应为:
\begin{equation}
    a_{ji} = \frac{1}{a_{ij}}
\end{equation}

对于准则层的7x7判断矩阵,通过查阅金融领域相关资料,我们先对准则层的各准则重要程度进行了标度,重要程度的排序结果为:$$\left[ in\_max,in\_sum,in\_avg,out\_max,out\_sum,out\_avg,ishot\right]=[1,6,4,2,7,5,3]$$

数值越大代表对应因素的重要程度越高。对于比较矩阵A中的某一项$A\left[i\right]\left[j\right]:A\left[i\right]\left[j\right]=x_i/x_j$,其中$x_i$,$x_j$即为之前排序的重要程度。进而得到准则层的判断矩阵A,如表\ref{tablejudge}。

\begin{table}[H]
    \centering
    \begin{tabular}{|c|ccccccc|}%一个c表示有一列,格式为居中显示(center)
    \hline
        A & $B_1$ & $B_2$ & $B_3$ & $B_4$ & $B_5$ & $B_6$ & $B_7$   \\
    \hline 
    $B_1$  &  1   &$\frac{1}{6}$  &$\frac{1}{4}$  &$\frac{1}{2}$  &$\frac{1}{7}$  &$\frac{1}{5}$  &$\frac{1}{3}$\\
    $B_2$  &  6   &1              &$\frac{3}{2}$  &3              &$\frac{6}{7}$  &$\frac{6}{5}$  &2\\
    $B_3$  &  4   &$\frac{2}{3}$  &1              &2              &$\frac{4}{7}$  &$\frac{4}{5}$  &$\frac{4}{3}$\\
    $B_4$  &  2   &$\frac{1}{3}$  &$\frac{1}{2}$  &1              &$\frac{2}{7}$  &$\frac{2}{5}$  &$\frac{2}{3}$\\
    $B_5$  &  7   &$\frac{7}{6}$  &$\frac{7}{4}$  &$\frac{7}{2}$  &1              &$\frac{7}{5}$  &$\frac{7}{3}$\\
    $B_6$  &  5   &$\frac{5}{6}$  &$\frac{5}{4}$  &$\frac{5}{2}$  &$\frac{5}{7}$  &1              &$\frac{5}{3}$\\
    $B_7$  &  3   &$\frac{1}{2}$  &$\frac{3}{4}$  &$\frac{3}{2}$  &$\frac{3}{7}$  &$\frac{3}{5}$  &1\\
    \hline
\end{tabular}
    \caption{准则层的判断矩阵}
    \label{tablejudge}
\end{table}
对于措施层的7个$302 \times 302$判断矩阵,,这里以第一个因素进项发票价税合计的最大值$in\_max$为例,其余的6个$302\times 302$判断矩阵可以通过分析另外6个因素求得。
首先是通过之前的数据预处理策略获得每一个企业的进项发票价税合计的最大值$in_max$的取值,结果如下

\begin{table}[H]
    
    \label{tablesymbol}
    \centering
    \begin{tabular}{|c|c|}   
    \hline
    企业代号 & 进项发票价税合计的最大值 \\
    \hline 
    $E124$  &  1109999.99\\
    $E125$  &  1109999.99\\
    $E126$  &  1082430.50\\
    $E127$  &  1029999.99\\
    $E128$  &  1000000.00\\
         
    $E421$  &    10290.00\\
    $E422$  &     8350.00\\
    $E423$  &    33800.00\\
    $E424$  &     9765.00\\
    $E425$  &     9912.00\\
    \hline
    \end{tabular}
    \caption{附件2中302家企业的进项发票价税合计的最大值}
    \label{table2}
\end{table}


\section{其它小功能}
\subsection{脚注}

利用 \verb|\footnote{具体内容}| 可以生成脚注\footnote{脚注可以补充说明一些东西}。

\subsection{无序列表与有序列表}

无序列表是这样的:
\begin{itemize}
    \item one
    \item two
    \item ...
\end{itemize}

有序列表是这样子的:
\begin{enumerate}
    \item one
    \item two
    \item ...
\end{enumerate}

\subsection{字体加粗与斜体}

如果想强调部分内容,可以使用加粗的手段来实现。加粗字体可以用 \verb|\textbf{加粗}| 来实现。例如: \textbf{这是加粗的字体。 This is bold fonts} 。

中文字体没有斜体设计,但是英文字体有。\textit{斜体 Italics}。

\section{参考文献与引用}

参考文献对于一篇正式的论文来说是必不可少的,在建模中重要的参考文献当然应该列出。\LaTeX{}在这方面的功能也是十分强大的,下面进介绍一个比较简单的参考文献制作方法。有兴趣的可以学习 \verb|bibtex| 或 \verb|biblatex| 的使用。

\LaTeX{}的入门书籍可以看《\LaTeX{}入门》\cite{liuhaiyang2013latex}。这是一个简单的引用,用 \verb|\cite{bibkey}| 来完成。要引用成功,当然要维护好 bibitem 了。下面是个简单的例子。

 

%参考文献
\begin{thebibliography}{9}%宽度9
    \bibitem[1]{liuhaiyang2013latex}
    刘海洋.
    \newblock \LaTeX {}入门\allowbreak[J].
    \newblock 电子工业出版社, 北京, 2013.
    \bibitem[2]{mathematical-modeling}
    全国大学生数学建模竞赛论文格式规范 (2020 年 8 月 25 日修改).
    \bibitem{3} \url{https://www.latexstudio.net}
\end{thebibliography}

\newpage
%附录
\begin{appendices}

\section{模板所用的宏包}
\begin{table}[htbp]
    \centering
    \caption{宏包罗列}
    \begin{tabular}{ccccc}
        \toprule
        \multicolumn{5}{c}{模板中已经加载的宏包} \\
        \midrule
        amsbsy & amsfonts & {amsgen} & {amsmath} & {amsopn} \\
        amssymb & amstext & {appendix} & {array} & {atbegshi} \\
        atveryend & auxhook & {bigdelim} & {bigintcalc} & {bigstrut} \\
        bitset & bm    & {booktabs} & {calc} & {caption} \\
        caption3 & CJKfntef & {cprotect} & {ctex} & {ctexhook} \\
        ctexpatch & enumitem & {etexcmds} & {etoolbox} & {everysel} \\
        expl3 & fix-cm & {fontenc} & {fontspec} & {fontspec-xetex} \\
        geometry & gettitlestring & {graphics} & {graphicx} & {hobsub} \\
        hobsub-generic & hobsub-hyperref & {hopatch} & {hxetex} & {hycolor} \\
        hyperref & ifluatex & {ifpdf} & {ifthen} & {ifvtex} \\
        ifxetex & indentfirst & {infwarerr} & {intcalc} & {keyval} \\
        kvdefinekeys & kvoptions & {kvsetkeys} & {l3keys2e} & {letltxmacro} \\
        listings & longtable & {lstmisc} & {ltcaption} & {ltxcmds} \\
        multirow & nameref & {pdfescape} & {pdftexcmds} & {refcount} \\
        rerunfilecheck & stringenc & {suffix} & {titletoc} & {tocloft} \\
        trig  & ulem  & {uniquecounter} & {url} & {xcolor} \\
        xcolor-patch & xeCJK & {xeCJKfntef} & {xeCJK-listings} & {xparse} \\
        xtemplate & zhnumber &       &       &  \\
        \bottomrule
    \end{tabular}%
    \label{tab:addlabel}%
\end{table}%

以上宏包都已经加载过了,不要重复加载它们。

\section{排队算法--matlab 源程序}

\begin{lstlisting}[language=matlab]
kk=2;[mdd,ndd]=size(dd);
while ~isempty(V)
[tmpd,j]=min(W(i,V));tmpj=V(j);
for k=2:ndd
[tmp1,jj]=min(dd(1,k)+W(dd(2,k),V));
tmp2=V(jj);tt(k-1,:)=[tmp1,tmp2,jj];
end
tmp=[tmpd,tmpj,j;tt];[tmp3,tmp4]=min(tmp(:,1));
if tmp3==tmpd, ss(1:2,kk)=[i;tmp(tmp4,2)];
else,tmp5=find(ss(:,tmp4)~=0);tmp6=length(tmp5);
if dd(2,tmp4)==ss(tmp6,tmp4)
ss(1:tmp6+1,kk)=[ss(tmp5,tmp4);tmp(tmp4,2)];
else, ss(1:3,kk)=[i;dd(2,tmp4);tmp(tmp4,2)];
end;end
dd=[dd,[tmp3;tmp(tmp4,2)]];V(tmp(tmp4,3))=[];
[mdd,ndd]=size(dd);kk=kk+1;
end; S=ss; D=dd(1,:);
 \end{lstlisting}

 \section{规划解决程序--lingo源代码}

\begin{lstlisting}[language=c]
kk=2;
[mdd,ndd]=size(dd);
while ~isempty(V)
    [tmpd,j]=min(W(i,V));tmpj=V(j);
for k=2:ndd
    [tmp1,jj]=min(dd(1,k)+W(dd(2,k),V));
    tmp2=V(jj);tt(k-1,:)=[tmp1,tmp2,jj];
end
    tmp=[tmpd,tmpj,j;tt];[tmp3,tmp4]=min(tmp(:,1));
if tmp3==tmpd, ss(1:2,kk)=[i;tmp(tmp4,2)];
else,tmp5=find(ss(:,tmp4)~=0);tmp6=length(tmp5);
if dd(2,tmp4)==ss(tmp6,tmp4)
    ss(1:tmp6+1,kk)=[ss(tmp5,tmp4);tmp(tmp4,2)];
else, ss(1:3,kk)=[i;dd(2,tmp4);tmp(tmp4,2)];
end;
end
    dd=[dd,[tmp3;tmp(tmp4,2)]];V(tmp(tmp4,3))=[];
    [mdd,ndd]=size(dd);
    kk=kk+1;
end;
S=ss;
D=dd(1,:);
 \end{lstlisting}
\end{appendices}

\end{document} 